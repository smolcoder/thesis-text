\chapter{Апробация} 
\label{approbation}

В данной главе будет описан подход к решению задачи в рекомендациях музыки социальной сети ``Одноклассники'', основанный на переходе
от трека к исполнителю, а также будут приведены результаты, показывающие улучшение рекомендаций после апробации данного подхода.

\section{Музыкальный сервис социальной сети “Одноклассники”}

Музыкальный сервис социальной сети ``Одноклассники'' доступен каждому зарегистрированному пользователю. Среди многочисленных возможностей, 
которые осуществляет сервис, имеются также рекомендации песен и исполнителей, формирование индивидуальных радиостанций. Устроены они на 
основе \emph{коллаборативной фильтрации}. То есть для определения вкусов пользователей и рекомендации им подходящей музыки анализируется 
их активность в музыкальном сервисе: что слушал человек, как долго, в каком порядке, как много раз, что из прослушанного было добавлено 
в собственные списки, какие треки слушались подряд и так далее. Затем на основе этой информации определяются \emph{похожие} треки и исполнители,
которые и попадают в списки рекомендаций. Например, если зайти на страницу с музыкой какого-либо исполнителя, то вниманию пользователя будут 
предложены не только его песни, но 16 похожих на него исполнителей.


\section{Проблема определения похожих исполнителей}
\section{Описание предлагаемого подхода}
\section{Результаты}
\section{Выводы}