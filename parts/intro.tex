\startprefacepage	

Тегирование музыки -- процесс описания песни специально подобранными словами (тегами), которые так или иначе характеризуют ее: жанр, эмоции, настроение, география и так далее.
Иными словами, тегирование музыки {{---}} мультиклассовая классификация, где в качестве классов представлены теги.
Посредством подобных словестных характеристик можно решать многие прикладные задачи. Например, поиск музыки по тегам, построение персонализированных музыкальных рекомендации.
Коммерческие рекомандательные системы такие, как Last.fm и Pandora, используют тегирование для решения этих задач. 

Однако, с ростом объема музыки процесс ручного сопоставления 
тегов каждой песне становится физически невозможным. Существуют различные масштабируемые подходы к тегированию, например, фолксономия (социальное тегирование), web-mining, машинное обучение. 
В частности, MIR-сообществом\footnote{MIR \ld Music Information Retrieval} был предложен подход \emph{автоматического} тегирования на основе внутреннего содержания музыки: 
метод анализирует акустические волны и автоматически сопоставляет теги. 
Основным преимуществом данного подхода является то, что для тегирования не требуется человек, так как процесс прослушивания песни человеком заменяется на анализ волн.
Впоследствии к данной задаче были применены многие алгоритмы машинного обучения, например, SVM, GMM, Boosting, алгоритмы ближайших соседей. 
В работе Mohamed Sordo приводится сравнение этих методов на различных музыкальных датасетах, а также отмечается превосходство алгоритмов ближайших соседей (k-NN, CBDC) над другими алгоритмами.

Целью данной работы является исследование применения алгоритмов ближайших соседей к русской музыке, а также исследование предложенного метода модификации этих алгоритмов, который
позволит увеличить точность тегирования.
Ключевой идеей данной модификации является переход от тегов трека к тегам его исполнителя с дальнейшим переходом обратно к тегам трека.