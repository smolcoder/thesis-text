\startconclusionpage

В данной работе было проведено исследование существующих алгоритмов ближайших соседей, 
а также был предложен новый подход к автоматическому тегированию \ld переход к от трека к исполнителю.
При данном переходе для тегирования конкретного трека используется не только информация о нем самом,
но и теги других треков того же исполнителя. Таким образом стало возможно исправлять отдельные плохо 
протегированные треки, увеличивая качество тегирования в целом.
Для проверки данного подхода были проведены эксперименты по тегированию русской музыки с использованием
экспертно составленной выборки. Данные эксперименты показали, что использование предложенного метода 
привело к заметному увеличению точности тегирования как в задаче сопоставления тегов треку, так и в задаче поиска треков
по тегу, практически во всех исследованных вариациях алгоритмов $k$-NN и CBDC.
Переход от тегов трека к тегам исполнителя был успешно применен для фильтрации похожих исполнителей в 
рекомендациях музыкального сервиса социальной сети ``Одноклассники''.

Таким образом, данная работа полностью удовлетворяет поставленным требованиям.

В качестве дальнейшей работы предлагается исследование перехода к исполнителю применительно к другим
алгоритмам тегирования, основанным, например, на SVM, GMM, нейронных сетях и так далее. Также возможен
поиск способа учитывать при переходе к исполнителю и индивидуальные особенности треков.

